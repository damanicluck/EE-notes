\chapter{This is for stuff im too lazy to organize right now}

\section{Miller's Theorem}

\begin{figure}[H]
    \centering
    \begin{circuitikz}
        \coordinate (start) at (0,0);
        \draw 
        (start) node[op amp] (amp) {}
        (amp.out) to [short, *-o] ++(1,0) node[label=right:$v_{out}$]{}
        (amp.out) 
            to [R, l=R] ++(-2,0)
            to [amp.-]
        % to [short, o-*] ++(1,0) node[label=left:S] at (S){}
        ;
    \end{circuitikz}
    % \begin{circuitikz}

    % \end{circuitikz}
\end{figure}

source: \href{https://eng.libretexts.org/Bookshelves/Electrical_Engineering/Electronics/Semiconductor_Devices_-_Theory_and_Application_(Fiore)/06%3A_Amplifier_Concepts/6.5%3A_Miller's_Theorem}{text later1}

\section{Second item}

source: 