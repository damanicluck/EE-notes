\section{Practice Problems}
\begin{enumerate}
    \item A uniform bar of $n$-type silicon of 2-$\mu$m length has a voltage of 1 V applied across it. If $N_D = 10^{16}$ \conc and $\mu_n = 1350$ cm\sq/V $\cdot$ s, find (a) the electron drift velocity, (b) the time it takes an electron to cross the 2-$\mu$m length, (c) the drift-current density, and the (d) drift current in the case that the silicon bar has a cross-sectional area of 0.25 \mun\sq.
    \begin{Ans}
        \begin{todo}
           \item TODO: finish out this question
        \end{todo}
    \end{Ans}

    \item A general relationship for the current density carried out by holes of density $p$ is $J = qpv$, where $q$ is the electronic charge and $v$ is the hole velocity.
    \begin{enumerate}
        \item Find the velocity of holes, $v(x)$, that are moving only by diffusion if they have a density distribution of $p(x) = p_0 e^{-x/l}$ . The electric field is zero.
        \item What would be the electric field that would lead to a hole drift velocity equal to that of the diffusion velocity in part(a)? Use Einstein's relation to answer this question.
        \item At 300 K, what is the value of $l$ to make the ecltric field in part (b) be 1000 V/cm?
    \end{enumerate}
\end{enumerate}